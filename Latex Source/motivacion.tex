\chapter{Motivación}
 La creciente demanda de energía eléctrica y la cada vez mayor contribución de las energías renovables al abastecimiento energético hacen necesario un cambio en la operación de las redes eléctricas. Las redes eléctricas deben evolucionar desde su tradicional característica de elemento conductor de la electricidad hasta convertirse en una infraestructura que no sólo permita el flujo de electrones, sino también el flujo de información entre los distintos participantes en el sistema eléctrico, convirtiéndose así en redes inteligentes (Smartgrids). Para conseguir el desarrollo completo de las redes inteligentes, es necesario flexibilizar la demanda de los clientes domésticos, a fin de y aprovechar su potencial de contribución a la estabilidad del sistema eléctrico. Sin embargo, los consumidores residenciales no podrán gestionar su demanda si no disponen de la infraestructura necesaria. 
 
 La Comisión Europea, en su marco estratégico en materia de clima y energía para el año 2020 [1] propone el objetivo prioritario para la eficiencia energética de ahorrar energía en un 20\% , y aquí se establece el punto de partida de este trabajo. Hasta ahora, las principales medidas se han basado en reducir el gasto de generación de la energía, fomentando el uso de fuentes de energía renovables o la fabricación de dispositivos de bajo consumo, pero las últimas tendencias dejan paso a una gestión del lado de la demanda, que en este nuevo escenario pasa a ser una parte activa. En la última década ha habido notables avances en este terreno, afianzándose el concepto de Hogar Digital y Smart Grid
 
  el término Smart Grid se utiliza para referirse a las redes de distribución eléctricas "inteligentes", donde la electricidad puede ser bidireccional, permitiendo a las viviendas convertirse también en pequeños productores de electricidad y no solo consumidores
 \section
 En la Unión Europea, el consumo energético ha estado tradicionalmente ligado al crecimiento económico, y aunque seguirá siendo uno de los motores de nuestra sociedad y economía, en el futuro el progreso debería ser con menos energía y menos costes, o al menos, haciendo un uso más eficiente de ella. Así, la Comisión Europea en su marco estrátegico en materia de clima y energía para el año 2020 propone el objetivo de mejorar la eficiencia energética.
 
 Hasta ahora, los principales avances se han fundamentado en reducir el gasto de generación de la energía, promover el uso de fuentes renovables o el uso de dispositivos con consumos cada vez más bajos. En los últimos años, aún sin olvidar las estrategias anteriores, la tendencia está cambiando. 