\chapter{Motivación}
 En la Unión Europea, el consumo energético ha estado tradicionalmente ligado al crecimiento económico, y aunque seguirá siendo uno de los motores de nuestra economía, en el futuro el progreso debería ser posible con menos costes y menos energía, o al menos, haciendo un uso más eficiente de los recursos. Así, la Comisión Europea, en su marco estratégico en materia de clima y energía, para el año 2020 propone el objetivo de mejorar la eficiencia energética.
 
 Hasta ahora, los principales avances en este tema se han fundamentado principalmente en reducir el gasto de generación de la energía, promover el uso de fuentes renovables o el uso de dispositivos con consumos cada vez más bajos. En los últimos años, aún sin olvidar las estrategias anteriores, la tendencia está cambiando. Se da paso a una gestión en el lado de la demanda, que se convierte en una parte activa de este nuevo escenario. 
 
 Con esta creciente demanda, la cada vez mayor contribución de las energías renovables y el aumento del auto-abastecimiento energético se hace necesario un cambio en la gestión de las redes eléctricas.Estas deben pasar de ser un mero elemento conductor de energía, a una infraestructura inteligente que no solo cumpla con los requisitos de distribución, sino que permita el flujo de información entre los distintos actores del sistema eléctrico. Transformándose, por tanto, en lo que se conoce como "Smart Grid".
 
 El creciente interés en las "Smart Grids" se puede comprobar con el gran número de investigaciones y equipos dedicados a ellas. Como puede ser......Lorem ipsum dolor sit amet, consectetur adipiscing elit. Aliquam efficitur sed est eu pretium. Fusce consectetur urna sed cursus posuere. Donec facilisis sit amet nisl blandit viverra. Vivamus at libero nulla. Phasellus sed maximus nisi. Maecenas vestibulum ante consequat enim sollicitudin pellentesque. Proin massa felis, ultrices at pulvinar scelerisque, ornare sit amet ipsum. Nam a fermentum elit.
 
 Donec tortor nibh, blandit sit amet molestie id, tempor sit amet mauris. Maecenas interdum rutrum eros, ut efficitur libero consectetur sed. Ut porttitor urna nec finibus sodales. Nulla neque orci, aliquet at nunc varius, elementum gravida lacus. Vivamus mattis, nisi nec semper tincidunt, augue nunc commodo urna, eget pellentesque sapien quam id nunc. Praesent fermentum nisi ac dolor convallis venenatis. Mauris pellentesque quis est vel efficitur.
 
 Integer id tellus luctus, elementum sem a, convallis tellus. Vestibulum sapien eros, tempor eget sagittis vel, tincidunt quis eros. Vivamus sodales tempus tellus at suscipit. In auctor justo id lacus porta cursus. Cras id efficitur dolor. Duis vitae turpis at ex sodales bibendum. Praesent leo est, aliquet pharetra suscipit nec, mattis et mi. Ut tempor, ante eu congue laoreet, diam justo rutrum arcu, sit amet molestie enim lectus vitae lacus. Aliquam viverra tempor ex non pharetra.
 
 Otro motivo por el que las "Smart Grids" seguirán creciendo en relevancia es la inminente revolución del coche eléctrico. Aunque su la penetración masiva no se espera en Europa hasta aproximadamente el 2030, actualmente ya suponen un porcentaje a considerar, y siguen creciendo. La sociedad ve en estos una clara alternativa de futuro, traen consigo un descenso en el consumo, y sobre todo una reducción en la contaminación ambiental.
 
 Con el coche eléctrico, toda la energía que ahora mismo proviene de derivados del petróleo se irá moviendo hacia energía eléctrica. Su recarga, generará una importante demanda para la que tendremos que estar preparados. Se necesitará un gran desplieguen de nuevos puntos de abastecimiento eléctrico. En este punto, las redes eléctricas inteligentes demostrarán su utilidad, ya que se requerirá gran flexibilidad y una adecuada gestión de la carga de la red. 
 
 Las "Smart Grids" ya están siendo desplegadas a nivel mundial y, especialmente a nivel europeo. Dentro de las posibles tecnologías adecuadas para estas redes, las tecnologías de tipo NB-PLC (Narrowband Power Line Communications) son las que más éxito tienen. Esto es así por las múltiples ventajas que ofrecen. Como por ejemplo, la infraestructura de comunicaciones ya está desplegada, el cable eléctrico, con lo que se reducen los gastos de despliegue. Prueba de ello es que según la orden IET/290/2012, en España el 100\% de los contadores eléctricos deberán ser contadores inteligentes para el 2018, y todos ellos usarán tecnologías de tipo NB-PLC. Aproximadamente el 50\% utilizará la especificación PRIME (PoweRline Intelligent Metering Evolution), estos son los desplegados por Iberdrola, Fenosa, EDP, etc...
 
 
 
 
 
 
 
 
 