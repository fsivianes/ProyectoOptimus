\chapter{Motivación}
 En la Unión Europea, el consumo energético ha estado tradicionalmente ligado al crecimiento económico, y aunque seguirá siendo uno de los motores de nuestra economía, en el futuro el progreso debería ser posible con menos costes y menos energía, o al menos, haciendo un uso más eficiente de los recursos. Así, la Comisión Europea, en su marco estratégico en materia de clima y energía, para el año 2020 propone el objetivo de mejorar la eficiencia energética.
 
 Hasta ahora, los principales avances en este tema se han fundamentado principalmente en reducir el gasto de generación de la energía, promover el uso de fuentes renovables o el uso de dispositivos con consumos cada vez más bajos. En los últimos años, aún sin olvidar las estrategias anteriores, la tendencia está cambiando. Se da paso a una gestión en el lado de la demanda, que se convierte en una parte activa de este nuevo escenario. 
 
 Con esta creciente demanda, la cada vez mayor contribución de las energías renovables y el aumento del auto-abastecimiento energético se hace necesario un cambio en la gestión de las redes eléctricas.Estas deben pasar de ser un mero elemento conductor de energía, a una infraestructura inteligente que no solo cumpla con los requisitos de distribución, sino que permita el flujo de información entre los distintos actores del sistema eléctrico. Transformándose, por tanto, en lo que se como "Smart Grid".
 